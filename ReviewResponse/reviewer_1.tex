\reviewer{Prof. Sumiko Miyata}


% ---------
% Comment 1
% ---------

\begin{revcomment}
  In your dissertation,
  you mentioned that there are other studies in imitation learning.
  I did not deeply understand the differences between existing imitation learning and your proposal.
  Moreover,
  in your slides,
  you didn't cite any imitation learning references.
  It is difficult to understand your novelty compared to existing studies.
  % You need to add differences in your slides and dissertation.
\end{revcomment}

\begin{revresponse}
  Thank you very much for your valuable comment.

  I would like to highlight the differences and novelty of my proposals compared to existing imitation learning (IL) studies.

  The existing IL studies mainly focus on training an agent to perform a single task well in a given domain.
  Although they achieved high performance on a single task under a specific domain,
  they are not able to generalize their learned behaviors to tackle future unseen situations.
  Moreover,
  when given a new task,
  those agents have to start the learning process again from the ground up,
  even if it has already learned a task that is related to and shares the same structure with the new one.
  The problem is referred to as generalization in IL and remains a fundamental challenge for modern IL.
  Therefore,
  the primary goal of my dissertation is to improve the generalization of IL agents.

  Unlike those approaches,
  In order to address the problem,
  I utilized adversarial learning to design objective functions for the agent's learning process.
  It allows the agent to effectively extract the underlying structure of the task.
  These extracted features help the agent adapt its learned knowledge to a new domain or new task effectively, resulting in an improvement in the agent's generalization ability as shown in the experimental results.

  These information was discussed in my dissertation on pages 3-4 and 16-18.
  I am really sorry for not including them in the presentation slides.
  The slides have been updated in order to clarify the novelty and contributions of my proposal.

  \begin{correction}
    No correction was made in the dissertation.
  \end{correction}
\end{revresponse}


% ---------
% Comment 2
% ---------

\begin{revcomment}
  How to collect expert demonstrations?
\end{revcomment}

\begin{revresponse}
  Thank you very much for your question.

  In the experiments, I considered six simulated tasks with varying difficulties:
  \begin{itemize}
    \item Easy: Pendulum and CartPole
    \item Medium: WindowOpen and WindowClose
    \item Hard: Door and Hammer
  \end{itemize}

  In order to train and adapt the proposed IL agents,
  expert demonstrations for all tasks must be provided.
  For the easy and medium tasks
  (i.e., Pendulum, CartPole, WindowOpen, and WindowClose),
  I utilized an RL method and trained it on each task in order to create an expert RL agent.
  After that,
  the demonstrations were collected by executing the trained expert RL agent in the simulated task a number of times.

  On the other hand,
  since the Door and Hammer tasks have a high difficulty level,
  I leveraged a publicly available dataset \cite{Task_Adroit}.
  The dataset was collected by humans using the Mujoco HAPTIX system \cite{Mujoco_HAPTIX}.
  The system is a combination of motion capture,
  physics simulation,
  and stereoscopic visualization to enable a user to interact with and manipulate virtual objects in the simulation.

  This information was discussed in my dissertation on pages 25-26.
  I also have added these details to the presentation slides.

  \printpartbibliography{Task_Adroit, Mujoco_HAPTIX}

  \begin{correction}
    No correction was made in the dissertation.
  \end{correction}

\end{revresponse}
