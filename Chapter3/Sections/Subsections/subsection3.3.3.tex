During the learning phase,
in order to learn domain-shared features between $\mathcal{E}$ and $\mathcal{L}$ domains,
the feature extractor $F$ and the generator $G$ are optimized to minimize the feature extractor loss $\mathfrak{L}_F$.
At the same time,
given a feature vector $\mathbf{f}_x$ of domain $x$,
we want to judge whether $\mathbf{f}_x$ is from $\mathcal{E}$ or $\mathcal{L}$ by minimizing the domain classification loss $\mathfrak{L}_{GAN}$.
This encourages domain-specific features to be captured by $F$.
Overall, the full objective function is:

\begin{align}
   & \underset{F, G}{max} \underset{D}{min} \mathfrak{L}(F, G, D) \\
   & \quad\text{subject to \:}
  \mathfrak{L}(F, G, D) = {\mathfrak{L}_{GAN}(G, D)} - \lambda\mathfrak{L}_F
\end{align}

The goal is to find a saddle point, where:

\begin{align}
  (\hat{F}, \hat{G}) & = \underset{F, G}{argmax}{\mathfrak{L}(F, G, \hat{D})}    \\
  \hat{D}            & = \underset{D}{argmin}{\mathfrak{L}(\hat{F}, \hat{G}, D)} \\
\end{align}

At the saddle point,
the $\hat{D}$ minimizes the domain classification loss.
The feature extractor $\hat{F}$ and the generator $\hat{G}$ minimize the distance between both domains (i.e. the features are shared between domains),
while maximizing the domain classification loss (i.e. the features are specific to each domain).
The parameter $\lambda$ controls the trade-off between domain-shared features and domain-specific features should be learned by $F$.

The learning algorithm of the proposed agent is outlined in Algorithm \ref{ch:DAIL:alg:ProposedModel}.

\begin{algorithm}
  \caption{\DAIL{}}
  \label{ch:DAIL:alg:ProposedModel}

  \begin{algorithmic}[1]
    \Input
    \Desc{$\mathcal{D}_\mathcal{E}$}{A set of expert demonstrations}
    \EndInput

    \State Randomly initialize feature extractor network $F$, generator $G$ and discriminator $D$
    \For {i = 0, 1, 2, ...}
    \State Sample an expert demonstration $\tau^i_\mathcal{E} \sim \mathcal{D}_\mathcal{E}$
    \State Update the parameters of feature extractor network $F$ with the gradient
    \[\mathbf{E}[
        \nabla_F log(D( \mathbf{f}_\mathcal{E} ))
      ] + \mathbf{E}[
        \nabla_F log(1 - D( \mathbf{f}_\mathcal{L} ))
      ] - \lambda \mathbf{E}[
        \nabla_F \left\|
        \mathbf{f}_\mathcal{E} - \mathbf{f}_\mathcal{L}
        \right\|
      ]
    \]
    \State Update the discriminator parameters with the gradient
    \[\mathbf{E}[
        \nabla_D log(D( \mathbf{f}_\mathcal{E} ))
      ] + \mathbf{E}[
        \nabla_D log(1 - D( \mathbf{f}_\mathcal{L} ))
      ]\]
    \State Update policy $\pi_{L}$ with the reward signal $r=-logD(\mathbf{f}_\mathcal{E})$
    \EndFor

    \Output
    \Desc{$\pi_{L}$}{Learned policy for learner domain}
    \EndOutput
  \end{algorithmic}
\end{algorithm}

