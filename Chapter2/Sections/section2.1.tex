Reinforcement learning (RL) is an approach to solving sequential decision-making tasks in complex and stochastic environments.
In RL,
the learner or the decision-maker is called the \textit{agent}.
The thing it interacts with,
including everything outside the agent,
is called the \textit{environment}~\cite{RL_Sutton2018}.
A key aspect of RL is that it tries to mimic how humans learn new things from interaction with the environment.
In other words,
the RL agent has to actively interact with the environment through \textit{trial and error} in order to gather experiences and find a sequence of actions that can solve a given task.
RL is one of three basic machine learning paradigms,
as depicted in Figure~\ref{ch:Background:fig:AI_RL_VennDiagram}.

\Figure{0.5\textwidth}{\FigsDir/AI_RL_VennDiagram.png}%
{Reinforcement learning is a machine learning approach to artificial intelligence.\label{ch:Background:fig:AI_RL_VennDiagram}}


\subsection{Episodic Finite-horizon Markov Decision Processes}
Consider a sequential decision-making problem in which an agent is faced with a task of influencing an environment through the actions it takes.
At each time step,
the agent observes the environment and must decide on which action to perform.
The agent affects the environment through these actions.
This action alters the environment and determines the immediate reward the agent receives.
The interaction between the agent and its environment is formalized as a \textit{Markov decision process} (MDP)~\cite{RL_Bellman1957}.
This dissertation focuses on the episodic finite-horizon MDP,
defined as follows.

\[\mathbb{M} = (\mathcal{S},\mathcal{A},P,R,\gamma,H)\]

\begin{itemize}
  \item The state space $\mathcal{S}$ is a finite set of states that the environment can have.
  \item The action space $\mathcal{A}$ is a finite set of actions the agent is allowed to take after observing a state.
  \item The transition function $P(\cdot|s,a)$ gives the probability of transition $P(s'|s,a) = \mathrm{Pr}(s^{t+1}=s'|s^t=s,a^t=a)$ at time step $t$ from state $s$ to state $s'$ under action $a$.
  \item The reward function $R: \mathcal{S} \times \mathcal{A} \to \mathbb{R}$ gives the immediate reward $R(s,a)$ for taking an action $a$ after observing state $s$.
  \item $\gamma \in [0,1)$ is the discount factor.
  \item The horizon $H \in \mathbb{N}$ is the number of time steps in each episode.
\end{itemize}

The decision rules that determine the agent's action for state $s$ is a policy,
denoted by $\pi(a|s)$.
In other words,
the policy $\pi(a|s)$ determines the agent's behaviors.

Figure \ref{ch:Background:fig:InteractionAgentEnv} illustrated the interaction between the agent and its environment.
At each time step $t$,
the agent observes state $s^t$ and takes an action $a^t$ according to its policy $\pi(a|s)$.
The environment transitions to $s^{t+1}$ according to the transition distribution $P(\cdot|s^t,a^t)$.
The agent then received a reward $r^t = R(s^t,a^t)$ and observes the state $s^{t+1}$ of the next time step $t+1$.
This interaction loop between the agent and environment continues for a total of $H$ time steps and generates a trajectory $\tau = (s^0,a^0,s^1,a^1,\dots,s^H,a^H)$

\Figure{0.7\textwidth}{\FigsDir/RL_AgentEnvInteraction.png}%
{The agent-environment interaction in Markov decision process.\label{ch:Background:fig:InteractionAgentEnv}}


\subsection {Reinforcement Learning in Episodic Fixed-horizon MDPs}
In an episodic fixed-horizon MDP setting,
an RL agent is assumed to

\begin{itemize}
  \item know the state space $\mathcal{S}$ and action space $\mathcal{A}$
  \item know the immediate reward $R(s,a)$ when taking action $a$ after observing state $s$
  \item does not know the transition function $P(\cdot|s,a)$
\end{itemize}

The agent can only learn about $P(\cdot|s,a)$ through interaction with the environment.
Given an MDP process,
the agent's goal is to find an optimal policy $\pi^*(a|s)$ that maximizes the expected discounted sum of rewards along trajectories $J(\pi)$.

\[
  J(\pi) = \mathbb{E}_\pi [\sum^H_{t=0} \gamma^t R(s^t,a^t)] = \mathbb{E}_\pi [\sum^H_{t=0} \gamma^t r^t]
\]

That is,
$\pi^*(a|s)=\argmax_\pi J(\pi)$.
The discount factor $\gamma$ values the immediate reward above delayed rewards.
The following subsection briefly reviews several approaches to finding $\pi^*(a|s)$ in RL.


\subsection{Reinforcement Learning Algorithms}
\Figure{\textwidth}{\FigsDir/RL_Algorithms.png}%
{Several examples of reinforcement learning algorithms.\label{ch:Background:fig:RL_Algorithms}}

An overview of RL algorithms is illustrated in Figure~\ref{ch:Background:fig:RL_Algorithms}.
RL algorithms can be divided into two main categories:

\begin{description}
  \item[Model-based algorithms]
        Model-based algorithms aim to find the environment dynamics model.
        In other words,
        it tries to establish a complete MDP by estimating the transition and reward functions.
        When the dynamics model is available,
        the problem of finding an optimal policy is called \textit{planning}.
        Several model-based algorithms include \textit{World models}~\cite{RL_Algo_WM},
        \textit{Imagination-Augmented Agents} (I2A)~\cite{RL_Algo_I2A},
        \textit{Model-Based Priors for Model-Free Reinforcement Learning} (MBMF)~\cite{RL_Algo_MBMF},
        \textit{Model-Based Value Expansion} (MBVE)~\cite{RL_Algo_MBVE},
        and \textit{AlphaZero}~\cite{RL_Algo_AlphaZero}.
  \item[Model-free algorithms]
        Model-free algorithms are data-driven and rely on trial-and-error experiences to find the optimal policy.
        Due to the difficulties of establishing a complete MDP,
        model-free algorithms have been the main focus of research compared to model-based algorithms.
        Several model-free algorithms include \textit{Policy Gradient}~\cite{},
        \textit{Asynchronous Advantage Actor-Critic} (A3C)~\cite{RL_Algo_A3C},
        \textit{Proximal Policy Optimization} (PPO)~\cite{RL_Algo_PPO},
        \textit{Trust Region Policy Optimization} (TRPO)~\cite{RL_Algo_TRPO},
        \textit{Deep Deterministic Policy Gradients} (DDPG)~\cite{RL_Algo_DDPG},
        \textit{Soft Actor-Critic} (SAC)~\cite{RL_Algo_SAC},
        \textit{Twin Delayed Deep Deterministic Policy Gradients} (TD3)~\cite{RL_Algo_TD3},
        \textit{Deep Q Neural Network} (DQN)~\cite{RL_Algo_DQN},
        \textit{C51}~\cite{RL_Algo_C51},
        \textit{Distributional Reinforcement Learning with Quantile Regression} (QR-DQN)~\cite{RL_Algo_QRDQN},
        and \textit{Hindsight Experience Replay} (HER)~\cite{RL_Algo_HER}.
\end{description}

There are two main approaches to training an RL agent with model-free algorithms:

\begin{description}
  \item[Policy optimization]
        Policy optimization seeks to directly learn $\pi^*(a|s)$ by applying gradient ascent on the objective $J(\pi)$.
  \item[Q-learning]
        Q-learning methods aim to discover the optimal policy by finding a policy that maximizes the action-value function $q_\pi(s,a)$.
        The $q_\pi(s,a)$ is defined as the expectation of cumulative discounted rewards.
        \[
          q_\pi(s,a) = \mathbb{E}_\pi [\sum^H_{t=0} \gamma^t r^t | s^t=s,a^t = a]
        \]
        The optimal action-value function $q_{\pi^*}(s,a)$ is the maximum action-value function over all policies.
        It indicates the maximum possible reward the agent can extract from the environment starting at state $s$ and taking action $a$.
        \[q_{\pi^*}(s,a)=\max_\pi q_\pi(s,a)\]
        In case $q_{\pi^*}(s,a)$ is known,
        the agent can decide which action to take to maximize the return.
        Thus,
        it can behave optimally in the MDP and therefore solve the MDP task.
\end{description}

Policy optimization methods are more stable than Q-learning methods since they directly optimize the policy using numerical optimization techniques such as gradient ascent.
In contrast,
Q-learning methods indirectly find the optimal policy by optimizing $q_\pi(s,a)$.
However,
Q-learning has the advantage of being more data efficient since it can reuse data more effectively than policy optimization.
Fortunately,
many algorithms have been designed utilizing both methods to carefully trade-off between the strengths and weaknesses of either policy optimization or Q-learning methods,
such as DDPG,
TD3,
and SAC.

