\added{RL has been primarily limited to
  tasks in which a well-defined reward function is given.
  However,
  the assumption of having access to a reward function is not feasible in complex tasks or the real-world environment.
  For example,
  consider the task of learning a good policy for autonomous driving.
  While a human driver is able to drive safely on roads,
  he/she may not be able to mathematically formulate a reward function that accurately represents good driving behaviors.
  Moreover,
  it may be impossible to design such a reward function given the dynamics of the environment
  (e.g.,
  stoplights,
  safety signs,
  traffic jams,
  pedestrians,
  etc.)
  Without a good reward function,
  RL is not practical for the autonomous driving problem.
}

Fortunately,
a good policy can still be learned by directly imitating demonstrations provided by an expert.
This approach to learning a policy by mimicking an expert's behaviors to accomplish a task is called imitation learning.
In IL,
the expert can be referred to as the teacher,
while the agent is the learner.
IL does not require the explicit design of a reward function.
Moreover,
since the expert demonstrations directly provide rich information regarding how to perform the task optimally,
IL typically requires fewer interactions with the environment than RL.

In imitation learning setting,
a task can be formulated as an MDP without a reward function,
i.e.,
$\mathbb{M}^- = \mathbb{M} \setminus R = (\mathcal{S},\mathcal{A},P,\gamma,H)$
Thus,
in an episodic fixed-horizon MDP setting,
an IL agent is assumed to

\begin{itemize}
  \item know the state space $\mathcal{S}$,
        action space $\mathcal{A}$
  \item does not know the immediate reward $R(s,a)$ when taking action $a$ after observing a state $s$
  \item does not know the transition distribution $P(\cdot|s,a)$ and the reward function $R$
\end{itemize}

Formally,
the IL agent is provided with a set of expert demonstration $\mathcal{D}_\mathcal{E} = \{ \tau^i_\mathcal{E} : i \in [1,N] \}$ that is collected by letting the expert interact in the task MDP $\mathbb{M}^-$.
Each $\tau^i_\mathcal{E} = \{ (s^t,a^t):t \in [0,H] \} = (s^0,a^0,\dots,s^H,a^H)$ denotes a single demonstration or an episode,
which is a sequence of state-action pairs.
The goal of the IL agent is to learn an optimal policy $\pi^*(a|s)$ against unknown reward function $R$,
given the expert demonstrations $\mathcal{D}_\mathcal{E}$.
Since the agent does not have access to immediate rewards during the learning process,
IL is considered to be more challenging compared to reinforcement learning.

There are two approaches to training an IL agent.

\begin{description}
  \item[Direct]
        This approach utilizes supervised learning to learn a policy from expert demonstrations.
        An example of this approach is \textit{Behavior cloning} (BC).

  \item[Indirect]
        This approach learns the unknown reward function using expert demonstrations and derives an optimal policy from it.
        An example of this approach is \textit{Inverse reinforcement learning} (IRL).
\end{description}

In the following subsections,
BC and IRL are introduced in detail.


\subsection{Behavior Cloning}
Behavioral cloning directly learns a policy using supervised learning on state-action pairs from expert demonstrations.
The IL agent is trained to reproduce the expert behaviors: for a given state,
the action taken by the agent must be the one taken by the expert.
In other words,
for any given expert state-action
pair $(s,a)$,
BC treat the state $s$ as the label and the action $a$ as the target.
Then,
IL becomes a classification or regression problem with state as the input and action as the output.

BC is a simple approach and does not require any interaction with the MDP,
but the learned policy often poorly generalizes.
The main reason for this is the assumption: supervised learning assumes that the state-action pairs are independent.
However,
in MDP,
an action in a given state induces the next state,
which breaks the previous assumption.
Moreover,
a mistake made by the agent can easily add up and put it into a state that the expert has never visited and the agent has never trained on.
In such states,
the behavior is undefined,
leading to catastrophic failures.
Therefore,
although the main advantage of BC is its simplicity,
it is unsuitable for tasks requiring long-term planning.


\subsection{Inverse Reinforcement Learning}
Inverse reinforcement learning is a different approach to imitation learning.
Its main idea is to learn a representation of the underlying reward function $R$ of the environment based on expert demonstrations.
IRL parameterizes the reward function $R$ as a linear combination of features:

\[
  R(s,a) = \mathbf{w}^T \phi(s,a)
\]

where $\mathbf{w} \in \mathbb{R}^n$ is a weight vector and $\phi(s,a):\mathcal{S}\times\mathcal{A}\to\mathbb{R}^n$ denotes a feature vector.

For a given feature map $\phi$,
the goal of IRL is to determining the optimal weights $\mathbf{w}$ that maximizes the expected discounted sum of rewards along trajectories $J(\pi)$.

\begin{align*}
  J(\pi) & = \mathbb{E}_\pi [\sum^H_{t=0} \gamma^t R(s^t,a^t)]                  \\
         & = \mathbf{w}^T \mathbb{E}_\pi [\sum^H_{t=0} \gamma^t  \phi(s^t,a^t)]
\end{align*}

It is important to note that,
the optimal policy $\pi^*$ always produce a better $J(\pi)$:

\begin{align*}
  J(\pi^*)                                                                                & \geq J(\pi)                                                             & ,\forall \pi \\
  \Leftrightarrow  \mathbf{w}^T \mathbb{E}_{\pi^*} [\sum^H_{t=0} \gamma^t  \phi(s^t,a^t)] & \geq \mathbf{w}^T \mathbb{E}_\pi [\sum^H_{t=0} \gamma^t  \phi(s^t,a^t)] & ,\forall \pi
\end{align*}

Given the above constraint,
it can be seen that $\mathbf{w} = 0$ satisfies the condition.
This problem is called reward ambiguity,
one of IRL's main challenges.
Besides,
for most expert's behavior,
there are many fitting reward functions.
Thus,
finding an optimal reward function can be quite challenging.
Despite that,
IRL can find a better policy than BC and can be applied in complex and long-term planning tasks.

